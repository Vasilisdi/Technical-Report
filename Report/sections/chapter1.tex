\chapter{Introduction}
\label{chap1}
{
Motivation and description. Points:
\begin{itemize}
	\item Importance of maintenance
	\item Predictive maintenance
	\item Object of my thesis
	\item Contribution of my thesis
\end{itemize}
}


Εδώ αυτή κάνουμε μια γενική περιγραφή του χώρου εφαρμογής της διπλωματικής. Αναφέρουμε τα χαρακτηριστικά του χώρου και καταλήγουμε στα γενικότερα προβλήματα που αντιμετωπίζει ο χώρος. Η συζήτηση των προβλημάτων θα πρέπει να προϊδεάζει τον αναγνώστη για το τι θα προσπαθήσει να αντιμετωπίσει η διπλωματική, χωρίς ακόμα να αναφερόμαστε συγκεκριμένα στο αντικείμενο της διπλωματικής.
  


\section{Object of the Diploma Thesis}
{
Elaboration on a system that predicts the failure of mechanical components based on historical data. This thesis focuses on resolving the issue of unplanned shutdowns due to corrective maintenance of machines that have bearings.
}
Εδώ αναφερόμαστε συγκεκριμένα στο τί θα κάνει η διπλωματική. Αναφέρουμε λεπτομερώς α) τα προβλήματα που θα λύσει (και που ήδη έχουν περιγραφεί γενικά στην προηγούμενη ενότητα), και β) πώς σκοπεύει να τα λύσει. 
Είναι σημαντικό κάποιος που θα διαβάσει την ενότητα αυτή να καταλάβει σε σημαντικό βαθμό τον σκοπό της διπλωματικής σας και τις τεχνικές δυσκολίες της, χωρίς να είναι αναγκαίο να δει όλα τα άλλα κεφάλαια. Η ενότητα αυτή θέλει πολύ προσοχή και καλύτερα να τη γράψετε αφού έχετε γράψει όλα τα υπόλοιπα κεφάλαια.

\subsection{The Concept of Predictive Maintenance}
{
Getting a good grasp on the upcoming required maintenance activities and being aware of what is of utmost importance is vital for the operation of a plant. t a grasp on
Periodical status checks of different machines is crucial for the maintenance engineers, since this could yield a clear understanding of what priority is at that instance of time. Therefore the prediction of an upcoming machine failure would constitute an invaluable tool.
\begin{itemize}
	\item Good condition readings
	\item Periodical readings
	\item Readings analysis
	\item online board
\end{itemize}
}

\subsection{Contribution}
Εδώ παραθέτουμε αριθμητικά συγκεκριμένες ενέργειες/λύσεις/μεθοδολογίες που παρουσιάζει η διπλωματική και λύνουν τα προβλήματα που υποσχεθήκαμε στην προηγούμενη ενότητα ότι θα λύσει η διπλωματική. Συνήθως η υποενότητα αυτή έχει την παρακάτω μορφή:

Η συνεισφορά της διπλωματικής συνοψίζεται ως εξής:
\begin{enumerate}
\item Μελετήθηκαν συστήματα κ.λ.π.
\item Υλοποιήθηκαν τρεις αλγόριθμοι υπολογισμού κ.λ.π.
\item Αξιολογήθηκε η επίδοση των αλγορίθμων και βρέθηκε ότι κ.λ.π.
\item Ενσωματώθηκαν οι αλγόριθμοι σε πρότυπο σύστημα κ.λ.π.
\item ...
\end{enumerate}


\section{Structure of the Thesis}

Εδώ περιγράφουμε τα κεφάλαια της διπλωματικής: μία πρόταση για το τί θα έχει  κάθε κεφάλαιο.Συνήθως η ενότητα αυτή έχει την παρακάτω μορφή (δεν θα σας πάρει πάνω από μία μεγάλη παράγραφο):

Εργασίες σχετικές με το αντικείμενο της διπλωματικής παρουσιάζονται στο Κεφάλαιο \ref{chap2}. Το Κεφάλαιο ... συζητά θέματα μοντελοποίησης. Στο Κεφάλαιο ... αναπτύσσουμε κ.λ.π. 

Για την τελική οργάνωση του κειμένου σας, συμβουλευθείτε τον επιβλέποντα της εργασίας.


