% Παράρτημα B

 
\chapter{Τίτλος 2ου Παραρτήματος\label{appendixB}}

Τα παραρτήματα περιλαμβάνουν συνοδευτικό, υποστηρικτικό υλικό (πίνακες, φωτογραφίες, ερωτηματολόγια, στατιστικά στοιχεία, αποδείξεις, περιγραφές  λογισμικών  προγραμμάτων,  παραδείγματα,  περιγραφές 
πολύπλοκων διαδικασιών, λίστα με πρωτογενή στοιχεία, λεπτομερής περιγραφή και προδιαγραφές εξοπλισμού, οδηγίες εγκατάστασης λογισμικού, κ.λπ.), ή αλλιώς ό,τι θεωρείται χρήσιμο να περιγραφεί, αλλά δεν συνηθίζεται να 
εντάσσεται μέσα στο κυρίως κείμενο της Εργασίας.  Στο κυρίως κείμενο της Εργασίας πρέπει να γίνονται οι κατάλληλες παραπομπές προς τα παραρτήματα, όπου το κείμενο σχετίζεται με υλικό που περιλαμβάνεται σε αυτά. Ένα παράρτημα, αναλόγως με το περιεχόμενό του, μπορεί να είναι ενιαίο, ή να χωρίζεται σε ενότητες.