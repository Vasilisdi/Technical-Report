%%
% Είδος: διπλωματική στα αγγλικά
\documentclass{uthece-thesis} 

% Πακέτα και ορισμοί που τυχόν χρειάζονται, ανάλογα με το κείμενο της διπλωματικής
\usepackage{algorithm}
\usepackage{algorithmic}

\usepackage[hyphens]{url}
\usepackage{multicol}
\usepackage{multirow}
\usepackage[table]{xcolor}
\usepackage{caption} 
\captionsetup[table]{skip=10pt}
\usepackage{tikz}
\usepackage{datetime}
\usepackage{ifthen}
\usepackage{hyperref} % Για ενεργά URLs


%αριθμός γενικού μητρώου: Μ011723007
%αριθμός μητρώου: 00578


% Πρόσθετοι ορισμοί, ανάλογα με το κείμενο της διπλωματικής
% typeset source code
\newcommand{\src}[1]{{\tt#1}}

% typeset a backslash
\newcommand{\bkslash}{\symbol{92}}

% Κύρια γλώσσα της διπλωματικής
% μην την τροποποιείτε
\ifgrthesisoption 
    \setlanguage{greek}
\else
    \setlanguage{american} 
\fi

%%%%%%%%%%%%%%%%%%%%%%%%%%%%%%%%%%%%%%%%%%%%%%%%%%%%%
%% THESIS INFO / ΠΛΗΡΟΦΟΡΙΕΣ ΔΙΠΛΩΜΑΤΙΚΗΣ
%%
% Τίτλος Διπλωματικής Εργασίας στα ελληνικά
	\title{Διερεύνηση Σχεδίασης Συστημάτων Ανάλυσης Δονήσεων Μηχανών}
% Τίτλος Διπλωματικής Εργασίας στα αγγλικά
	\titleEng{Investigation of Machine Vibration Analysis System Design}
% Ονοματεπώνυμο φοιτητή στα ελληνικά
	\edef\authorname{Βασίλειος Δημητρίου}
	\edef\secauthorname{}
% Ονοματεπώνυμο φοιτητή στα αγγλικά 
	\edef\authornameEng{Vasileios Dimitriou}
	\edef\secauthornameEng{}
% Ονοματεπώνυμο Επιβλέποντα Καθηγητή στα ελληνικά
	\supervisor{Καρακωνσταντής Γεώργιος}
% Ονοματεπώνυμο Επιβλέποντα Καθηγητή στα αγγλικά
	\supervisorEng{Karakonstantis Georgios}
% Φύλλο επιβλέποντα στα ελληνικά 
% Μην το τροποποιείτε για διπλωματική στα αγγλικά
	\edef\supervisorMaleFemale{Επιβλέπων/πουσα}
% Στοιχεία επιτροπής στα αγγλικά
    \edef\supervisortitle{Rank/position of Supervisor}
    \edef\supervisoraffiliation{Department of Electrical and Computer Engineering, University of Thessaly}
    \edef\memberonename{Name Surname of Member 1}
    \edef\memberonetitle{Rank/position of Member 1}
    \edef\memberoneaffiation{Department/Institution of Member 1}
    \edef\membertwoname{Name Surname of Member 2}
    \edef\membertwotitle{Rank/position of Member 2}
    \edef\membertwoaffiation{Department/Institution of Member 2}    
% Ημερομηνία υποβολής διπλωματικής
    \newdate{docdate}{04}{02}{2022}
% Ημερομηνία έγκρισης διπλωματικής
    \newdate{docapprovedate}{24}{02}{2022}
% format ημερομηνίας - μην το τροποποιείτε
    \newdateformat{mydate}{\THEDAY-\THEMONTH-\THEYEAR} \mydate
% Τόπος και έτος διπλωματικής στα ελληνικά
	\edef\thesisPlaceDate{Μήνας \getdateyear{docdate}}
% Τόπος και έτος διπλωματικής στα αγγλικά
	\edef\thesisPlaceDateEng{Month \getdateyear{docdate}}
%


%%%%%%%%%%%%%%%%%%%%%%%%%%%%%%%%%%%%%%%%%%%%%%%%%%%%


\begin{document}

\frontmatter
\maketitle


%%%%%%%%%%%%%%%%%%%%%%%%%%%%%%%%%%%%%%%%%%%%%%%%%%%%%
%% OPTIONAL MATERIAL / ΠΡΟΑΙΡΕΤΙΚΕΣ ΕΝΟΤΗΤΕΣ
%%
% Κατάλογος Σχημάτων προαιρετικά
%	\listoffigures % σε σχόλια για μη εμφάνιση
% Κατάλογος Πινάκων προαιρετικά
%	\listoftables % σε σχόλια για μη εμφάνιση
% Συντομογραφίες - Αρκτικόλεξα - Ακρωνύμια προαιρετικά
%	% Συντομογραφίες - Αρκτικόλεξα - Ακρωνύμια

\newcommand{\abbrev}[2]{#1 \> #2\\ }
\begin{abbreviations}

\begin{tabbing}
%ta 'a' rythmizoun to platos ton dyo stilon
  aaaaaaaaaaaaaaaaa \= aaaaaaaaaaaaaaaaaaaaaa\kill
  \abbrev{βλπ}{βλέπε}
  \abbrev{κ.λπ.}{και λοιπά}
  \abbrev{κ.ο.κ}{και ούτω καθεξής}
  \abbrev{ΤΕΙ}{Τεχνολογικό Εκπαιδευτικό Ίδρυμα}
  \abbrev{BPF}{Band Pass Filter}
\end{tabbing}
\end{abbreviations} % σε σχόλια για μη εμφάνιση 
%
%%%%%%%%%%%%%%%%%%%%%%%%%%%%%%%%%%%%%%%%%%%%%%%%%%%%


\mainmatter
%%%%%%%%%%%%%%%%%%%%%%%%%%%%%%%%%%%%%%%%%%%%%%%%%%%%%
%% CHAPTERS / ΚΕΦΑΛΑΙΑ 
%%
    \chapter{Introduction}
\label{chap1}
{
Motivation and description. Points:
\begin{itemize}
	\item Importance of maintenance
	\item Predictive maintenance
	\item Object of my thesis
	\item Contribution of my thesis
\end{itemize}
}


Εδώ αυτή κάνουμε μια γενική περιγραφή του χώρου εφαρμογής της διπλωματικής. Αναφέρουμε τα χαρακτηριστικά του χώρου και καταλήγουμε στα γενικότερα προβλήματα που αντιμετωπίζει ο χώρος. Η συζήτηση των προβλημάτων θα πρέπει να προϊδεάζει τον αναγνώστη για το τι θα προσπαθήσει να αντιμετωπίσει η διπλωματική, χωρίς ακόμα να αναφερόμαστε συγκεκριμένα στο αντικείμενο της διπλωματικής.
  


\section{Object of the Diploma Thesis}
{
Elaboration on a system that predicts the failure of mechanical components based on historical data. This thesis focuses on resolving the issue of unplanned shutdowns due to corrective maintenance of machines that have bearings.
}
Εδώ αναφερόμαστε συγκεκριμένα στο τί θα κάνει η διπλωματική. Αναφέρουμε λεπτομερώς α) τα προβλήματα που θα λύσει (και που ήδη έχουν περιγραφεί γενικά στην προηγούμενη ενότητα), και β) πώς σκοπεύει να τα λύσει. 
Είναι σημαντικό κάποιος που θα διαβάσει την ενότητα αυτή να καταλάβει σε σημαντικό βαθμό τον σκοπό της διπλωματικής σας και τις τεχνικές δυσκολίες της, χωρίς να είναι αναγκαίο να δει όλα τα άλλα κεφάλαια. Η ενότητα αυτή θέλει πολύ προσοχή και καλύτερα να τη γράψετε αφού έχετε γράψει όλα τα υπόλοιπα κεφάλαια.

\subsection{The Concept of Predictive Maintenance}
{
Getting a good grasp on the upcoming required maintenance activities and being aware of what is of utmost importance is vital for the operation of a plant. t a grasp on
Periodical status checks of different machines is crucial for the maintenance engineers, since this could yield a clear understanding of what priority is at that instance of time. Therefore the prediction of an upcoming machine failure would constitute an invaluable tool.
\begin{itemize}
	\item Good condition readings
	\item Periodical readings
	\item Readings analysis
	\item online board
\end{itemize}
}

\subsection{Contribution}
Εδώ παραθέτουμε αριθμητικά συγκεκριμένες ενέργειες/λύσεις/μεθοδολογίες που παρουσιάζει η διπλωματική και λύνουν τα προβλήματα που υποσχεθήκαμε στην προηγούμενη ενότητα ότι θα λύσει η διπλωματική. Συνήθως η υποενότητα αυτή έχει την παρακάτω μορφή:

Η συνεισφορά της διπλωματικής συνοψίζεται ως εξής:
\begin{enumerate}
\item Μελετήθηκαν συστήματα κ.λ.π.
\item Υλοποιήθηκαν τρεις αλγόριθμοι υπολογισμού κ.λ.π.
\item Αξιολογήθηκε η επίδοση των αλγορίθμων και βρέθηκε ότι κ.λ.π.
\item Ενσωματώθηκαν οι αλγόριθμοι σε πρότυπο σύστημα κ.λ.π.
\item ...
\end{enumerate}


\section{Structure of the Thesis}

Εδώ περιγράφουμε τα κεφάλαια της διπλωματικής: μία πρόταση για το τί θα έχει  κάθε κεφάλαιο.Συνήθως η ενότητα αυτή έχει την παρακάτω μορφή (δεν θα σας πάρει πάνω από μία μεγάλη παράγραφο):

Εργασίες σχετικές με το αντικείμενο της διπλωματικής παρουσιάζονται στο Κεφάλαιο \ref{chap2}. Το Κεφάλαιο ... συζητά θέματα μοντελοποίησης. Στο Κεφάλαιο ... αναπτύσσουμε κ.λ.π. 

Για την τελική οργάνωση του κειμένου σας, συμβουλευθείτε τον επιβλέποντα της εργασίας.



    \chapter{Theoretical Background}
\label{chap2}
\section{Fundamentals of Bearing Fault Frequencies}

\section{Introduction to System Architecture}
% Overview + diagram

\section{Algorithms and Technologies}
% Summary of FFT, data flow, etc.

\section{Technologies}
\subsection{Embedded Systems \& IoT}
\subsubsection{ADXL335 Sensor}
Analog signal from sensor breakout to arduino. To elaborate on signal transferring
\subsubsection{Arduino Microcontroller}
UART serial communication sending data to RPi. To elaborate on this protocol
\subsubsection{Raspberry Pi 4}
RPi receiving data using a script and posing HTTP Post data to a supabase api. Elaborate on this


\subsection{Signal Processing}
\subsubsection{Fast Fourier Transform (FFT)}
Theory, implementation

\subsection{Backend \& Database}
\subsubsection{Supabase}
\begin{itemize}  
	\item PostgreSQL. 
	\item APIs
\end{itemize}  
\subsubsection{Vercel and Next.js}


    %\include{chapter...} % προσθέστε όσα χρειάζεται
    \chapter{Conclusions}
\label{chap_last}
Εδώ εξηγούμε ότι θα συνοψίσουμε την μελέτη που εκπονήθηκε στα πλαίσια της διπλωματικής.
{
\begin{itemize}
\item summarize how it decides if a bearing is defective
\item how it collects data
\item how data is analysed and at what stages it is analysed
\end{itemize}
}

\section{Σύνοψη και συμπεράσματα}
Εδώ συνοψίζουμε τα αποτελέσματα της διπλωματικής και περιγράφουμε τα συμπεράσματα που προέκυψαν, αρνητικά και θετικά. Επιβεβαιώνουμε την συνεισφορά της διπλωματικής στα προβλήματα που αναφέραμε στην εισαγωγή.
{

}

\section{Μελλοντικές επεκτάσεις}
Εδώ δίνουμε ιδέες για επέκταση της διπλωματικής.
{

}

%
%%%%%%%%%%%%%%%%%%%%%%%%%%%%%%%%%%%%%%%%%%%%%%%%%%%%


%\backmatter % μην ενεργοποιείτε την εντολή
% Βιβλιογραφία - Αναφορές
	\bibliography{sections/references}


%%%%%%%%%%%%%%%%%%%%%%%%%%%%%%%%%%%%%%%%%%%%%%%%%%%%%
%% APPENDICES (optional) / ΠΑΡΑΡΤΗΜΑΤΑ (ΠΡΟΑΙΡΕΤΙΚΑ)
%%

% Εάν υπάρχουν περισσότερα από ένα Παραρτήματα
% είναι ενεργή η ακόλουθη εντολή
% και ανενεργή για μεμονωμένο Παράρτημα
    \include{sections/appendices}

% Μην τροποποιήσετε τις ακόλουθες εντολές
\ifgrthesisoption
    \renewcommand{\chaptername}{Παράρτημα}
\else
    \renewcommand{\chaptername}{Appendix}
\fi

% Πριν το κάθε Παράρτημα (για περισσότερα από ένα), 
% ρυθμίζεται η κατάλληλη αλφαβητική αρίθμηση
    \renewcommand{\thechapter}{A} 
  	\include{sections/appA}	
% %	
    \renewcommand{\thechapter}{B}
    \include{sections/appB}

% Εάν υπάρχει μόνο ένα Παράρτημα,
% μπαίνουν σε σχόλια οι παραπάνω γραμμές 
% και ρυθμίζεται αρίθμηση χωρίς γράμμα
%   \renewcommand{\thechapter}{} 
%   \renewcommand{\thesection}{\arabic{section}}
%
% Ακολουθεί αρχείο μεμονωμένου Παραρτήματος
% (διαφορετική δομή από appA/appΒ)
%   \include{single_app} 

%
%%%%%%%%%%%%%%%%%%%%%%%%%%%%%%%%%%%%%%%%%%%%%%%%%%%%


\end{document}